\documentclass[10pt,conference,compsocconf,letterpaper]{IEEEtran}

\usepackage{amsfonts}
\usepackage[cmex10]{amsmath}
%\usepackage{subfigure}
\usepackage{amssymb}
\usepackage[american]{babel}
%\usepackage[small,it]{caption}
\usepackage{cite}
\usepackage{graphicx}
\usepackage[frenchlinks]{hyperref}
\usepackage{microtype}
\usepackage{relsize}
\usepackage{times}
%\usepackage{xspace}

\title{Conference paper}

\author{%
\IEEEauthorblockN{Mark Stillwell}
\IEEEauthorblockA{Department of Engineering Computing\\
Cranfield University\\
Cranfield, UK\\
Email: m.stillwell@cranfield.ac.uk}}

\interdisplaylinepenalty=2500

\begin{document}

\maketitle
\date{}
\pagestyle{empty}

\begin{abstract}
Abstract Text
\end{abstract}

\section{Introduction}

%% Context and setup

%% General Description of Contribution

%% Brief Contrast with Existing Solutions to Further Justify

%% Bulleted List of Specific contributions / claims
In this paper we: 
\begin{itemize}
\item demonstrate or prove new and useful things
\end{itemize}

%% Layout of paper
This paper is organized as follows: In section...

\section{Problem Description / Formalization}
\label{sec.problem}

\section{Experiment}

%% Description of Goals

%% Description and Justification of Methodology

\section{Results}
\label{sec.results}

\section{Related Work}
\label{sec.related}

Ideally this should be section N-1 and should show why what we propose is
different from what others have done.

For some projects the need to set up and establish backround means that this
should be section 2 or 3.

\section{Conclusion}
\label{sec.conclusion}

%% Summary of Paper, including Major Conclusions

%% Future Work

\bibliographystyle{IEEEtran}
\bibliography{strings,journals,scheduling,procs}

\end{document}
